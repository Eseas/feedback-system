\documentclass[11pt]{article}

\usepackage[utf8]{inputenc} % `utf8` option to match Editor encoding
\usepackage[L7x,T2A]{fontenc}
\usepackage[utf8]{inputenc}
\usepackage[lithuanian]{babel}

\usepackage{graphicx} 
\usepackage{float}

\usepackage[top=2cm, bottom=2cm, left=3cm, right=1.5cm]{geometry}

\usepackage{booktabs} % in the preamble
\usepackage{tabu}

%%% Fix for L7x encoding
\makeatletter
\expandafter\let\csname L7x-cmd\endcsname\@changed@cmd
\makeatother

\begin{document}
\begin{titlepage}
    \centering
    \vfill
    {\Large%\bfseries
    	VILNIAUS UNIVERSITETAS\\
    	MATEMATIKOS IR INFORMATIKOS FAKULTETAS\\
    	PROGRAMŲ SISTEMŲ KATEDRA\\
		\vskip2cm
        {\Huge Iteracija nr. 1\\
        Strategija\\}
        \vskip2cm
        WhySoft komanda:\\
        Kazimieras Senvaitis\\
        Viktorija Sujetaitė\\
        Lukas Mikelionis\\
        Justinas Bukas\\
        Marius Sukarevičius\\
    }    
    \vskip2cm
    \includegraphics[width=4cm]{logo.jpg} % also works with logo.pdf
    \vfill
    Vilnius, 2017
    \vfill
\end{titlepage}

\section{Iteracijos}
Sistema bus kuriama iteraciniu modeliu, per 3-is iteracijas:\\
\begin{itemize}
	\item Iteracija nr. 1: nuo 2017-03-01 iki 2017-03-31 (4 sav.)
	\item Iteracija nr. 2: nuo 2017-04-01 iki 2017-04-23 (3 sav.)
	\item Iteracija nr. 3: nuo 2017-04-24 iki 2017-05-14 (3 sav.)
	\item Laiko rezervas: nuo 2017-05-15 iki 2017-05-31 (2 sav.) (pridavimo data dar tiksliai nesutarta)
\end{itemize}

\section{Koncepcinis projektavimas}
\textbf{Produkto kūrimas}

V. Sujetaitė dirbs su išorine sąsaja (angl. front-end).
Likę nariai dirbs su vidine dalimi (angl. back-end)\\

Projektuosime sistemos dalis taip, kad galima būtų visiems dirbti vienu metu nesukialiant problemų su sujungimo prašymais (angl. merge requests)\\
\\
\textbf{Produkto kūrimas}
Sistemos moduliai bus išskirti taip:
\begin{itemize}
	\item Naudotojų valdymo modulis
	\item Apklausų kūrimo modulis (didelis ir sudėtingas)
	\item Apklausų atsakymų surinkimo modulis
	\item Ataskaitų modulis (didelis)
\end{itemize}

\section{Rizikos}
Rizika - \underline{įtaką turintis įvykis}, kuris gali įvykti ar neįvykti.
\
Identifikuotos rizikos ir numatytas jų valdymas pavaizduotas lentelėje (1 lentelė)

\begin{table}[H]
\centering
\caption{Rizikos ir jų valdymas}
\label{table:ta}
\noindent\begin{tabu} to \textwidth {|X|X|X|}
   \toprule
   Rizika & Reaktyvus valdymas & Proaktyvus valdymas  \\
   \bottomrule
   Viktorija ir Marius gali susirgti & Perduoti darbą kitiems nariams & Nepervargti, valgyti sveiką vitamingą maistą \\
   \hline
   Lukas ir Viktorija gali nesusprėti ir mokytis ir dirbi. Lukas - 0.5 etato, Viktorija - 0.75. & Perduoti darbą kitiems nariams & - \\
   \hline
   Komanda gali susipykti tarpusavyje. & persiskirstyti roles & elgtis mandagiai, kalbėti apie problemas\\
   \hline
   Komandos nariai gali imti ignoruoti komandinius pokalbius Slack'e. & Iš naujo susitikti ir kalbėti apie Slack naudojimą & Facebook'e kylančias pokalbius nedelsiant nukreipti į slack. Skatinti naudoti Slack appsą mobiliąjame ir leisti programą background'e \\
   \hline
   Gali sugęsti kompiuteriai & dirbti MIFe, skolintis kompiuterį & apsvarstyti, iš kur gautumėte kompiuterį\\
   \hline
   Komandos nariai gali susirasti kitų veiklų (darbas, projektai ir pnš.) & Žeminti sistemos kokybę, perduoti darbus kitiems & nuosekliai dirbti, bendradarbiauti, kad būtų įdomu ir nesinorėtų keisti veiklos \\
   \hline
   Gali nepavykti dirbti komandiškai & persiskirstyti roles, išsiaiškinti problemas ir bandyti jas spręsti & nuosekliai dirbti, kalbėti apie problemas, nebūti kategoriškais\\
   \hline
   Gali būti, kad nepavyks pasirinkti kartu derančių technologijų & Pereiti prie kitų technologijų ir pradėti iš naujo & Pasikonsultuoti su dėstytojais dėl technologijų ir naudoti rekomenduojamas\\
   \hline
\end{tabu}

\end{table}

\section{Pakartotinas panaudojimas}
Abstrakcijos repanaudojimui bus naudojamos tik jau numatytose plečiamose dalyse. Visa kita kursime negalvodami apie pakartotinį panaudojimą.

\section{Metrikos}
Kiekvienas komandos narys matuos:
\begin{itemize}
	\item laiką praleistą programuojant ir studijuojant technologijos informaciją.
	\item laiką praleistą rašant dokumentaciją.
	\item padarytų svarbių defektų skaičių (kai defektas matomas iš vartotojo pusės ir neaišku, kur klaida kode)
\end{itemize}
Komanda matuos:
\begin{itemize}
	\item Stebėsime ir išsirinksime mums aktualią statistiką pateikiamą YouTrack įrankio. (Kol kas menkai mokame naudotis tuo įrankiu)
	\item Veikiausiai svarbiausia metrika bus suplanuoto ir realaus laiko skirtumas.
\end{itemize}

\section{Darbų valdymas}
Žinojimui kas ką daro naudosime YouTrack pritaikytą Scrum procesui.

\section{Konfigūracijos valdymas}
\begin{itemize}
	\item Kodo ir dokumentų versijavimui bus naudojamas git.mif.vu.lt.
	\item Naudosime Java EE 7 platformą.
	\item Kitas technologijas naudosime naujausios versijos.
\end{itemize}
	
\end{document}